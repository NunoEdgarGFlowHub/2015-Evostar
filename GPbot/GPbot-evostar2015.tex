
%%%%%%%%%%%%%%%%%%%%%%% file typeinst.tex %%%%%%%%%%%%%%%%%%%%%%%%%
%
% This is the LaTeX source for the instructions to authors using
% the LaTeX document class 'llncs.cls' for contributions to
% the Lecture Notes in Computer Sciences series.
% http://www.springer.com/lncs       Springer Heidelberg 2006/05/04
%
% It may be used as a template for your own input - copy it
% to a new file with a new name and use it as the basis
% for your article.
%
% NB: the document class 'llncs' has its own and detailed documentation, see
% ftp://ftp.springer.de/data/pubftp/pub/tex/latex/llncs/latex2e/llncsdoc.pdf
%
%%%%%%%%%%%%%%%%%%%%%%%%%%%%%%%%%%%%%%%%%%%%%%%%%%%%%%%%%%%%%%%%%%%


\documentclass[runningheads,a4paper]{llncs}

\usepackage{amssymb}
\setcounter{tocdepth}{3}
\usepackage{graphicx}

\usepackage{url}
\newcommand{\keywords}[1]{\par\addvspace\baselineskip
\noindent\keywordname\enspace\ignorespaces#1}

\begin{document}

\mainmatter  % start of an individual contribution

% first the title is needed
\title{It's time to stop: investigating termination conditions in games}

% a short form should be given in case it is too long for the running head
\titlerunning{It's time to stop}

% the name(s) of the author(s) follow(s) next
%
% NB: Chinese authors should write their first names(s) in front of
% their surnames. This ensures that the names appear correctly in
% the running heads and the author index.
%
\author{Non A. Me%
\thanks{NoInstitute}}
%
\authorrunning{Me, N}
% (feature abused for this document to repeat the title also on left hand pages)

% the affiliations are given next; don't give your e-mail address
% unless you accept that it will be published
\institute{No Institute}

%
% NB: a more complex sample for affiliations and the mapping to the
% corresponding authors can be found in the file "llncs.dem"
% (search for the string "\mainmatter" where a contribution starts).
% "llncs.dem" accompanies the document class "llncs.cls".
%

\maketitle


\begin{abstract}
Determining the end of the evolutionary process in the design of a game bot is not as easy as it would seem a priori. The first issue is that there is no optimum, because most games make sure that there is no optimum way of creating a bot that defeats all enemies. Then, in most games fitness is noisy, so that even if you use a certain level of fitness as a termination condition, this is going to change every generation it is evaluated. Finally, if you use external enemies and a certain level of victories over them the bots will be overfitted to those particular enemies. Even as any of these conditions is not perfect, there is the possibility of combining several of them as termination conditions so that bots that are as optimal as possible (and in as general a context as it is possible) are found; also, if possible, in a minimal amount of time and evaluations. In this paper we will examine several ways of determining the end of an evolutionary game bot design process, determining the capabilities of every one of them and, eventually, selecting one for future designs. We will use a game, Planet Wars, in which we already have some experience, but the conclusions to our study could, in principle, be extended to any RTS or even any game with non-deterministic match results.
\keywords{Games, RTS, evolutionary algorithms, coevolution}
\end{abstract}


\section{Introduction}

The termination condition in Evolutionary Algorithms (EA) is a key factor in the experimental setups, as it affects the algorithm's performance in... Usual termination conditions are a fixed number of generations (or evaluations), a fixed amount of time where the best solution is not improved, or distance to the optimum, among others.

This issue is very evident in the scope of the noisy and combat-based problems for player generation. In this area there are no optimum, as the opponents are not known before the EA run. To solve this, usually previously-known opponents are used in the fitness function. However, this can generate over-fitted solutions to only beat the used enemy.

The termination condition chosen can be applied to any game with stochastic behavior and combat-based results. In this paper, Planet Wars has been chosen in the experimental setup, as it is a simplified RTS game (only one type of resource, one type of attack and one type of unit).

\section{State of the Art}

\section{Conclusions}

\section*{Acknowledgements}

Hidden for double-blind review

\bibliographystyle{unsrt}
\bibliography{geneura}

\end{document}
