
%%%%%%%%%%%%%%%%%%%%%%% file typeinst.tex %%%%%%%%%%%%%%%%%%%%%%%%%
%
% This is the LaTeX source for the instructions to authors using
% the LaTeX document class 'llncs.cls' for contributions to
% the Lecture Notes in Computer Sciences series.
% http://www.springer.com/lncs       Springer Heidelberg 2006/05/04
%
% It may be used as a template for your own input - copy it
% to a new file with a new name and use it as the basis
% for your article.
%
% NB: the document class 'llncs' has its own and detailed documentation, see
% ftp://ftp.springer.de/data/pubftp/pub/tex/latex/llncs/latex2e/llncsdoc.pdf
%
%%%%%%%%%%%%%%%%%%%%%%%%%%%%%%%%%%%%%%%%%%%%%%%%%%%%%%%%%%%%%%%%%%%


\documentclass[runningheads,a4paper]{llncs}

\usepackage{amssymb}
\setcounter{tocdepth}{3}
\usepackage{graphicx}

\usepackage{url}
\newcommand{\keywords}[1]{\par\addvspace\baselineskip
\noindent\keywordname\enspace\ignorespaces#1}

\begin{document}

\mainmatter  % start of an individual contribution

% first the title is needed
\title{It's time to stop: investigating termination conditions in evolution of game bots}

% a short form should be given in case it is too long for the running head
\titlerunning{It's time to stop}

% the name(s) of the author(s) follow(s) next
%
% NB: Chinese authors should write their first names(s) in front of
% their surnames. This ensures that the names appear correctly in
% the running heads and the author index.
%
\author{Non A. Me%
\thanks{NoInstitute}}
%
\authorrunning{Me, N}
% (feature abused for this document to repeat the title also on left hand pages)

% the affiliations are given next; don't give your e-mail address
% unless you accept that it will be published
\institute{No Institute}

%
% NB: a more complex sample for affiliations and the mapping to the
% corresponding authors can be found in the file "llncs.dem"
% (search for the string "\mainmatter" where a contribution starts).
% "llncs.dem" accompanies the document class "llncs.cls".
%

\maketitle


\begin{abstract}
Determining the end of the evolutionary process in the design of a
game bot is not as easy as it would seem a priori. The first issue is
that there is no optimum, because most games make sure that there is
no optimum way of creating a bot that defeats all enemies. Then, in
most games fitness is noisy, so that even if you use a certain level
of fitness as a termination condition, this is going to change every
generation it is evaluated. Finally, if you use external enemies and a
certain level of victories over them the bots will be overfitted to
those particular enemies. Even as any of these conditions is not
perfect, there is the possibility of combining several of them as
termination conditions so that bots that are as optimal as possible
(and in as general a context as it is possible) are found; also, if
possible, in a minimal amount of time and evaluations. In this paper
we will examine several ways of determining the end of an evolutionary
game bot design process, determining the capabilities of every one of
them and, eventually, selecting one for future designs. We will use a
game, Planet Wars, in which we already have some experience, but the
conclusions to our study could, in principle, be extended to any RTS
or even any game with non-deterministic match results. 
\keywords{Games, RTS, evolutionary algorithms, coevolution}
\end{abstract}

\section{Introduction}

The termination condition in Evolutionary Algorithms (EA) is a key
factor in the experimental setups since it affects the performance of
the algorithm solution and the amount of resources devoted to it. The
usual termination conditions are a fixed number of generations (or
evaluations), which is related to a fixed {\em computing cycles}
budget for doing the experiments or a fixed amount of generations in
which the best solution is not improved, or distance to the optimum,
among others.

Finding the optimal termination condition is essential in the search for
bots that fight in a particular setup. The usual solution is to give
it as much as you have got, which is fair if you are designing a bot % add some reference to former articles where we did this. - JJ
to participate in a competition. However, from a methodological point
of view, if you want to compare different approaches to design a bot
it has to be done in conditions that are as similar as possible, which
might make {\em as much as you have got} a bit short on
details. Methods that are going to be compared need to have an
independent stopping criterium, since using for any method the same
time or the same number of evaluations is not usually fair. 
 
In the case of games, this issue is even more evident in the scope of
the noisy and combat-based methods used for player generation. In this
area there is no optimum (short from managing to obtain a bot that, as
in Sun Tzu's ``Art of war'', is able to beat the oponent without
fighting), as the opponents are not known before the EA is running. To
solve this, usually previously-known opponents are used in the fitness
function and an ``optimum'' is reached when a certain number of
victories are obtained. However, this can generate over-fitted
solutions that only beat the used enemy. % add references here. -JJ

The termination condition chosen can be applied to any game with
stochastic behavior and combat-based results. In this paper, Planet
Wars has been chosen in the experimental setup, as it is a simplified
RTS game (only one type of resource, one type of attack and one type
of unit).

Summarizing, our objective in this paper is to study different 
termination conditions to find the method that converges to good 
solutions (the best, if possible). To measure the quality of each 
method, we will use the time needed to obtain a solution and the quality of that solution as metric.

The rest of the paper is organized as follows. Next section
established the state of the art in the use of termination conditions
for this kind of evolutionary algorithms in which the optimum is not
known and/or is noisy. Next in Section \ref{sec:met} we will present the different termination
conditions we will be checking in this work and how they contribute to
prove which one has better behavior in this context; we will also
sketch the problem we are using in this particular paper, evolution of
bot for Planet Wars. Results will be presented in Section
\ref{sec:met}, followed by the conclusions and a discussion of the
implication of the obtained results.

\section{State of the Art}

\section{Methodology and experimental setup}
\label{sec:met}

\section{Results}
\label{sec:res}

\section{Conclusions}

\section*{Acknowledgements}

Hidden for double-blind review

\bibliographystyle{unsrt}
\bibliography{geneura,references}

\end{document}
