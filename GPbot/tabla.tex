%% LyX 2.0.6 created this file.  For more info, see http://www.lyx.org/.
%% Do not edit unless you really know what you are doing.
\documentclass[a4paper,runningheads]{llncs}
\usepackage[latin1]{inputenc}
\usepackage{amsmath}
\usepackage{amssymb}

\makeatletter

%%%%%%%%%%%%%%%%%%%%%%%%%%%%%% LyX specific LaTeX commands.
\special{papersize=\the\paperwidth,\the\paperheight}

%% Because html converters don't know tabularnewline
\providecommand{\tabularnewline}{\\}

%%%%%%%%%%%%%%%%%%%%%%%%%%%%%% User specified LaTeX commands.
%%%%%%%%%%%%%%%%%%%%%%% file typeinst.tex %%%%%%%%%%%%%%%%%%%%%%%%%
%
% This is the LaTeX source for the instructions to authors using
% the LaTeX document class 'llncs.cls' for contributions to
% the Lecture Notes in Computer Sciences series.
% http://www.springer.com/lncs       Springer Heidelberg 2006/05/04
%
% It may be used as a template for your own input - copy it
% to a new file with a new name and use it as the basis
% for your article.
%
% NB: the document class 'llncs' has its own and detailed documentation, see
% ftp://ftp.springer.de/data/pubftp/pub/tex/latex/llncs/latex2e/llncsdoc.pdf
%%%%%%%%%%%%%%%%%%%%%%%%%%%%%%%%%%%%%%%%%%%%%%%%%%%%%%%%%%%%%%%%%%%%





\usepackage{multirow}\usepackage{rotating}\usepackage{subfigure}%\usepackage{subfig}
\usepackage{url}

\newcommand{\keywords}[1]{\par\addvspace\baselineskip
\noindent\keywordname\enspace\ignorespaces#1}

\providecommand{\tabularnewline}{\\}



\makeatother

\begin{document}
\begin{tabular}{|c|c|c|c|c|c|c|c|}
\hline 
\multirow{2}{*}{SC}  & \multicolumn{3}{c|}{Absolute} & \multicolumn{3}{c|}{Relative} & \multirow{2}{*}{V} \tabularnewline
\cline{2-7} 
 & G  & S  & R  & G  & S  & V  & \tabularnewline
\hline 
\hline 
NG\_030.0  & 30.00  & 16.31  & 1.00  & 1.00  & 1.00  & 1.00  & 45.92 \tabularnewline
NG\_050.0  & 50.00  & 17.80  & 1.00  & 1.70  & 1.09  & 1.15  & 52.72 \tabularnewline
NG\_100.0  & 100.00  & 19.21  & 1.00  & 3.37  & 1.18  & 1.25  & 57.25 \tabularnewline
NG\_200.0  & 200.00  & 20.25  & 1.00  & 6.70  & 1.24  & 1.27  & 58.39 \tabularnewline
\hline 
AO\_1.0  & 8.83  & 13.46  & 1.00  & 0.29  & 0.83  & 0.78  & 35.89 \tabularnewline
AO\_1.5  & 10.33  & 14.07  & 1.00  & 0.34  & 0.86  & 0.76  & 34.83 \tabularnewline
AO\_2.0  & 14.61  & 14.93  & 1.00  & 0.49  & 0.92  & 0.83  & 38.08 \tabularnewline
AO\_2.5  & 17.17  & 15.30  & 1.00  & 0.57  & 0.94  & 0.87  & 39.92 \tabularnewline
\hline 
RT\_n/02  & 2.00  & 10.20  & 1.00  & 0.07  & 0.63  & 0.45  & 20.58 \tabularnewline
RT\_n/04  & 5.47  & 12.21  & 1.00  & 0.18  & 0.75  & 0.61  & 28.17 \tabularnewline
RT\_n/08  & 78.64  & 18.16  & 1.00  & 2.62  & 1.11  & 1.11  & 50.94 \tabularnewline
RT\_n/16  & 248.21  & 21.34  & 0.66  & 8.27  & 1.31  & 1.37  & 62.92 \tabularnewline
\hline 
FT\_20.0  & 55.08  & 20.62  & 1.00  & 1.84  & 1.26  & 1.18  & 54.22 \tabularnewline
FT\_22.0  & 127.56  & 22.65  & 1.00  & 4.25  & 1.39  & 1.27  & 58.25 \tabularnewline
FT\_24.0  & 276.71  & 24.39  & 0.77  & 9.22  & 1.50  & 1.38  & 63.39 \tabularnewline
FT\_26.0  & 378.88  & 26.45  & 0.22  & 12.63  & 1.62  & 1.63  & 74.75 \tabularnewline
\hline 
FI\_03.0  & 10.31  & 13.34  & 1  & 0.34  & 0.82  & 0.65  & 30.00 \tabularnewline
FI\_07.0  & 24.56  & 15.54  & 1  & 0.82  & 0.95  & 0.90  & 41.39 \tabularnewline
FI\_10.0  & 35.47  & 16.50  & 1  & 1.18  & 1.01  & 1.04  & 47.94 \tabularnewline
FI\_15.0  & 52.22  & 17.56  & 1  & 1.74  & 1.08  & 1.15  & 53.00 \tabularnewline
\hline 
\end{tabular}

$\in[0,1]$
\end{document}
