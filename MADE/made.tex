
%%%%%%%%%%%%%%%%%%%%%%% file typeinst.tex %%%%%%%%%%%%%%%%%%%%%%%%%
%
% This is the LaTeX source for the instructions to authors using
% the LaTeX document class 'llncs.cls' for contributions to
% the Lecture Notes in Computer Sciences series.
% http://www.springer.com/lncs       Springer Heidelberg 2006/05/04
%
% It may be used as a template for your own input - copy it
% to a new file with a new name and use it as the basis
% for your article.
%
% NB: the document class 'llncs' has its own and detailed documentation, see
% ftp://ftp.springer.de/data/pubftp/pub/tex/latex/llncs/latex2e/llncsdoc.pdf
%
%%%%%%%%%%%%%%%%%%%%%%%%%%%%%%%%%%%%%%%%%%%%%%%%%%%%%%%%%%%%%%%%%%%


\documentclass[runningheads,a4paper]{llncs}

\usepackage{amssymb}
\setcounter{tocdepth}{3}
\usepackage{graphicx}

\usepackage{url}
\newcommand{\keywords}[1]{\par\addvspace\baselineskip
\noindent\keywordname\enspace\ignorespaces#1}

\begin{document}

\mainmatter  % start of an individual contribution

% first the title is needed
\title{Creator of worlds: analysing the parameters that generate massive and coherent fiction environments}

% a short form should be given in case it is too long for the running head
\titlerunning{Short title}

% the name(s) of the author(s) follow(s) next
%
%
\author{Non A. Me%
\thanks{NoInstitute}}
%
\authorrunning{Me, N}
% (feature abused for this document to repeat the title also on left hand pages)

% the affiliations are given next; don't give your e-mail address
% unless you accept that it will be published
\institute{No Institute}

%
% NB: a more complex sample for affiliations and the mapping to the
% corresponding authors can be found in the file "llncs.dem"
% (search for the string "\mainmatter" where a contribution starts).
% "llncs.dem" accompanies the document class "llncs.cls".
%

\maketitle


\begin{abstract}
Generating fictional worlds according to some specific requisites
presents the problem that it is impossible to know the optimum for the
particular fitness function designed at the same time it creates a
vast search space for the evolutionary algorithm parameters that it
needs. In this paper we will try to find a methodology for finding the
best parameters of the evolutionary algorithm in the design of
fictional worlds. This evolutionary design includes running, to
completion, a world and assigning a fitness to it, so it is a problem
with a very complex fitness landscape. That is why, in order to
optimize resources given to evolution and also have some guarantee
that the final result will be close to optimal, we systematically
study the values of different parameters to try and find a set of
rules that will, once you have the fitness function, help you set a
reduced range of values o even a single value that will, with a
reduced computation budget, help you find an optimal world.

\keywords{Games, Plot, evolutionary algorithms, content generation, literature}
\end{abstract}

\section{Introduction}

In the very competitive cultural industry, writers rack their minds in
order to generate interesting fictional worlds, mainly for the
creation of richer environments in modern videogames.  

\section{State of the Art}


\section{Methodology and experimental setup} %Issue #12
\label{sec:met}

In this section we will show how to systematize the different steps of the proposed method to generate a fictional world. Initially, the fitness should be decided considering the desired behaviour of the agents that populate the world (for example, a given quantity of some archetype). Also, the number of profiles is...  Then, the characteristics of this world are also decided taking into account the expected... For example, adding enough days to allow the archetypes generation. Finally, the interesting plot points are extracted from the log using... 

As the usage of different number of profiles generate different individual sizes, and therefore, different convergence times... TERMINAR DE JUSTIFICAR CRITERIO DE PARADA

In this work, we propose three scenarios:
\begin{itemize}
\item Scenario A: CUENTO O LO QUE SEA Tiene 1 solo arquetipo
\item Scenario B: CUENTO O LO QUE SEA TIene 2 arquetipos
\item Scenario C: Tiene 5 arquetipos
\end{itemize}

\subsection{Fitness deduction}

The first step requires to develop a fitness function that takes into account the quantity of the desired archetypes. Three quantities will be defined: low, average, and...

In this paper, three fitness will be studied, related with the previously explained scenarios:
\begin{itemize}
\item Function 1: for the scenario A...
\item Function 2: for the scenario B...
\item Function 3: for the scenario C...
\end{itemize}

Number of profiles: the number of profiles affect the generated fitness as...

\subsection{System optimization}

Once the fitness function has been decided, several parameters related with the world and the GA will be studied

The parameters of the world affect  JUSTIFICAR ...


Number of days: 128, 256, 512


Finally, the parameters of the GA also affect the execution of... the population size ... 64, 128, 256 JUSTIFICAR

The rest of the parameters will be the classic... 1/n... JUSTIFICAR 


\subsection{Converting logs to stories}

% 12 pages is the limit. 
\section{Results}
\label{sec:res}

\section{Conclusions}

\section*{Acknowledgements}

Hidden for double-blind review

\bibliographystyle{unsrt}
\bibliography{geneura,references}

\end{document}
