
%%%%%%%%%%%%%%%%%%%%%%% file typeinst.tex %%%%%%%%%%%%%%%%%%%%%%%%%
%
% This is the LaTeX source for the instructions to authors using
% the LaTeX document class 'llncs.cls' for contributions to
% the Lecture Notes in Computer Sciences series.
% http://www.springer.com/lncs       Springer Heidelberg 2006/05/04
%
% It may be used as a template for your own input - copy it
% to a new file with a new name and use it as the basis
% for your article.
%
% NB: the document class 'llncs' has its own and detailed documentation, see
% ftp://ftp.springer.de/data/pubftp/pub/tex/latex/llncs/latex2e/llncsdoc.pdf
%
%%%%%%%%%%%%%%%%%%%%%%%%%%%%%%%%%%%%%%%%%%%%%%%%%%%%%%%%%%%%%%%%%%%


\documentclass[runningheads,a4paper]{llncs}

\usepackage{amssymb}
\setcounter{tocdepth}{3}
\usepackage{graphicx}

\usepackage{url}
\newcommand{\keywords}[1]{\par\addvspace\baselineskip
\noindent\keywordname\enspace\ignorespaces#1}

\begin{document}

\mainmatter  % start of an individual contribution

% first the title is needed
\title{Creator of worlds: analysing the parameters that generate massive and coherent fiction environments}

% a short form should be given in case it is too long for the running head
\titlerunning{Short title}

% the name(s) of the author(s) follow(s) next
%
%
\author{Non A. Me%
\thanks{NoInstitute}}
%
\authorrunning{Me, N}
% (feature abused for this document to repeat the title also on left hand pages)

% the affiliations are given next; don't give your e-mail address
% unless you accept that it will be published
\institute{No Institute}

%
% NB: a more complex sample for affiliations and the mapping to the
% corresponding authors can be found in the file "llncs.dem"
% (search for the string "\mainmatter" where a contribution starts).
% "llncs.dem" accompanies the document class "llncs.cls".
%

\maketitle


\begin{abstract}
Generating fictional worlds according to some specific requisites
presents the problem that it is impossible to know the optimum for the
particular fitness function designed at the same time it creates a
vast search space for the evolutionary algorithm parameters that it
needs. In this paper we will try to find a methodology for finding the
best parameters of the evolutionary algorithm in the design of
fictional worlds. This evolutionary design includes running, to
completion, a world and assigning a fitness to it, so it is a problem
with a very complex fitness landscape. That is why, in order to
optimize resources given to evolution and also have some guarantee
that the final result will be close to optimal, we systematically
study the values of different parameters to try and find a set of
rules that will, once you have the fitness function, help you set a
reduced range of values o even a single value that will, with a
reduced computation budget, help you find an optimal world.

\keywords{Games, Plot, evolutionary algorithms, content generation, literature}
\end{abstract}

\section{Introduction}

In the very competitive cultural industry, writers rack their minds in
order to generate interesting fictional worlds, mainly for the
creation of richer environments in modern videogames. In order to make
generation of these fictional worlds and the stories within them truly efficient and massive,
several methods have been proposed. One of them, called MADE
\cite{garcia14my} finds ``interesting'' character stories by running
an evolutionary algorithm that optimizes a fitness function designed
from the archetypes that best describe the world the client wants.

MADE works as follows: every individual in an evolutionary algorithm
is described by a chromosome that represents one (or several kinds of)
finite state automaton that are set to live in a simulated
environment. The evolutionary algorithm was run with standard
parameters and {\em sensible} default values were given to several
non-evolutionary parameters such as the number of profiles needed to
obtain the desired result and or the size of the simulated
world. There was no attempt to optimize these parameters, since the
main intention seemed to be to have a prototype that allowed the
authors to have an acid test of their proposed approach.

That paper hinted at the fact that parameters such as the number of
{\em profiles} (different kind of finite state automaton present in
the world) had a big influence on the outcome, to the point of making
it possible or not. However, other parameters such as the number of
simulated days the world is running might also have some influence.

In this paper we will use that open source simulator and look at it
from the evolutionary point of view so that we can check what kind of
influence they have in the outcome. We will do a experimental setup
that will test different parameters, and eventually find a series of
rules that will help the users of that framework to set the best
values for their simulation. 

The rest of the paper is organized as follows...

\section{State of the Art}


\section{Methodology and experimental setup}
\label{sec:met}


% 12 pages is the limit. 
\section{Results}
\label{sec:res}

\section{Conclusions}

\section*{Acknowledgements}

Hidden for double-blind review

\bibliographystyle{unsrt}
\bibliography{geneura,references}

\end{document}
